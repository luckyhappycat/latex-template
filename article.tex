%!TEX program = xelatex
\documentclass[10pt,onecolumn,a4paper]{article}
\usepackage{ctex} % 支持中文
\usepackage{enumerate} % 项目编号
%\usepackage{enumitem}
\usepackage{geometry} % 设置页边距
\geometry{left = 2.2cm, right=2.2cm, top = 2.5cm, bottom=2.5cm}
\usepackage{amssymb} % symbol
\usepackage{amsthm} % proof
%\usepackage{courier} % 代码字体
\usepackage{graphicx,subfigure} % figures
\usepackage{xcolor,mdframed} % mdframed
\usepackage{amsmath}
\usepackage{fancyhdr} % 添加页眉页脚
\usepackage{titlesec}
\titleformat*{\section}{\centering\bf\large} %设置章节字体
\usepackage{indentfirst} % 首行缩进
\setlength{\parindent}{2em} % 设置首行缩进两字符
\usepackage{breqn}
\renewcommand\d{\mathop{}\!\mathrm{d}}
\usepackage{multirow}

\newtheorem{theorem}{Theorem}
\renewcommand{\proofname}{\emph{\textbf{Proof}}}
%\definecolor{mycolor}{RGB}{200,200,200}
\definecolor{mycolor}{RGB}{192,192,192} % 设置代码阴影

\cfoot{\thepage}

% 代码
%\usepackage{listings}
\usepackage{fontspec}
\setmonofont[Mapping={}]{DejaVu Sans Mono} %英文引号之类的正常显示,相当于设置英文字体,windows下用Consolas字体,Linux下用DejaVu Sans Mono字体
%\setmonofont[Mapping={}]{Consolas} %英文引号之类的正常显示,相当于设置英文字体
\setmonofont{DejaVu Sans Mono}
%\setsansfont{Consolas} %设置英文字体 Monaco, Consolas,  Fantasque Sans Mono,Linux下用DejaVu Sans Mono字体
%\setmainfont{Consolas} %设置文章主体部分的英文字体

\setcounter{tocdepth}{2} % 设置目录深度

\begin{document}
\title{文档标题}
\author{作者}
%\today
\maketitle % 显示文档标题 作者 时间

\tableofcontents % 生成标题目录
%\listoftables % 生成表格目录
%\listoffigures % 生成图片目录
%\clearpage
%\newpage % 换页


\section{第一章标题}
\subsection{第一章第一节标题}
一级标题和二级标题:
\begin{mdframed}[backgroundcolor=mycolor,hidealllines=true]
\begin{verbatim}
\section{} 表示一级标题,自动对标题进行编号
\section*{} 表示一级标题,不对标题进行编号
\subsection{} 表示二级标题,自动对标题进行编号
\subsection*{} 表示二级标题,不对标题进行编号
\end{verbatim}
\end{mdframed}

书写代码的环境,设置背景颜色为黑色,不显示行号:
\begin{mdframed}[backgroundcolor=mycolor,hidealllines=true]
\begin{verbatim}
\ begin{mdframed}[backgroundcolor=mycolor,hidealllines=true]
\ begin{verbatim}
代码内容
\ end{verbatim}
\ end{mdframed}
%注意这里为了避免歧义,在反斜杠\后加了空格
\end{verbatim}
\end{mdframed}

插入图片:
\begin{mdframed}[backgroundcolor=mycolor,hidealllines=true]
\begin{verbatim}
\begin{figure}
  \centering % 设置图片居中显示
  \includegraphics[width=5in]{./当前目录下的图片如pic.jpg} % 设置了图片宽度
  \caption{图片标题用于在图片下显示}\label{图片标签用于引用}
\end{figure}
\end{verbatim}
\end{mdframed}

默认使用首行缩进,如果不希望缩进,那么需要使用:
\begin{mdframed}[backgroundcolor=mycolor,hidealllines=true]
\begin{verbatim}
\noindent
\end{verbatim}
\end{mdframed}

项目编号,默认起始编号为1.
\begin{mdframed}[backgroundcolor=mycolor,hidealllines=true]
\begin{verbatim}
\begin{enumerate}
    \item 这里默认起始编号为1.
\end{enumerate}
\end{verbatim}
\end{mdframed}

项目编号,默认起始编号为(1)
\begin{mdframed}[backgroundcolor=mycolor,hidealllines=true]
\begin{verbatim}
\begin{enumerate}[(1)]
\item
\end{enumerate}
\end{verbatim}
\end{mdframed}


项目编号,默认起始编号为(a)
\begin{mdframed}[backgroundcolor=mycolor,hidealllines=true]
\begin{verbatim}
\begin{enumerate}[(a)]
\item
\end{enumerate}
\end{verbatim}
\end{mdframed}


项目编号,默认起始编号为指定数字:
\begin{mdframed}[backgroundcolor=mycolor,hidealllines=true]
\begin{verbatim}
\begin{enumerate}
\setcounter{enumi}{4}
  \item 设置默认起始编号为5.
\end{enumerate}
\end{verbatim}
\end{mdframed}


\begin{table}[!hbp]
  \centering
  \begin{tabular}{|c|c|c|c|c|c|c|}
  \hline
  % after \\: \hline or \cline{col1-col2} \cline{col3-col4} ...
  项目1 & 2 & 3 & 4 & 5 & 6 & 7 \\ \hline
  项目2 & 8 & 9 & 10 & 11 & 12 & 13 \\ \hline
  项目3 & 一 & 二 & 三 & 四 & 五 & 六 \\ \hline
\end{tabular}
  \caption{表格名称}\label{表格标签}
\end{table}
如表\ref{表格标签}所示:


\begin{mdframed}[backgroundcolor=mycolor,hidealllines=true]
\begin{verbatim}
\begin{table}[!hbp]
  \centering
  \begin{tabular}{|c|c|c|c|c|c|c|}
  \hline
  % after \\: \hline or \cline{col1-col2} \cline{col3-col4} ...
  项目1 & 2 & 3 & 4 & 5 & 6 & 7 \\ \hline
  项目2 & 8 & 9 & 10 & 11 & 12 & 13 \\ \hline
  项目3 & 一 & 二 & 三 & 四 & 五 & 六 \\ \hline
\end{tabular}
  \caption{表格名称}\label{表格标签}
\end{table}
如\ref{表格标签}所示:
\end{verbatim}
\end{mdframed}


\end{document}